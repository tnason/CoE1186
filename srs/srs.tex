\documentclass[12pt]{article}

\usepackage[margin=1in]{geometry}
\usepackage{titlesec}

\usepackage[T1]{fontenc}
\usepackage{pslatex}

\begin{document}

\title{Software Requirement Specifications}
\author{Authors...}
\date{}
\maketitle

\section{Introduction}
     \subsection{Purpose}
     Purpose of the SRS and specify the audience of the SRS.

     \subsection{Scope}
     Identify the software products to be produced by name. Explain what the product will/will not do. Describe the application of software being specified; relevent benefits, objectives, and goals. Consistent with high level specs.

     \subsection{Definitions}
     Define all terms, acronyms and abbreviations. May refer to appendixes or other docs.

     \subsection{References}
     Complete list of docs refered to by the SRS. ID docs by title/report num/ date/ publisher. Specify sources from which ref can be obtained.

     \subsection{Overview}
     Describe the rest of the SRS. Explain how SRS is organized.

\section{Overall description}
Describe teh general factors taht affect the product and its requirements. Does not state specific requirements. Provide background for those requirements that are defined in section 3 and clarify them.

     \subsection{Product perspective}
     Put the product into perspective with other related products. State if product is independent/ totally self contained. If product is component of larger system, relate the requirements to the larger system and identify interfaces between system and the software.

          \subsubsection{System Interface}
	  List each system interface and identify the functionality of the software requirements and the interface description to match the system.
	  
	  \subsubsection{User Interface}
	  Logical characteristics of each interface between the software product and its users. Include configuration characteristics neccessary to accomplish the software requirements.
	  
	  Include aspects of optimizing the interface with the person who must use the system. Maybe list of do's and don'ts on how the system will appear to the user. Long/short error messages. Must be verifiable. 
	  
	  \subsubsection{Hardware interfaces}
	  Specify logical characteristics of each interface between the software and the hardware components. Include configuration characteristics. Cover what devices are supported, how they are supported and protocols.
	  
	  \subsubsection{Software interfaces}
	  Specify use of other required software products and interfaces with other application systems. For each required software product, provide: Name, mnemonic, specification number, version number and source.

	  For each interface: discuss the purpose of interfacing software as related to this product. Define the interface in terms of message content and format. 

	  \subsubsection{Communication interfaces}
	  Specify the various interfaces to communications sucha as local network protocols etc.

	  \subsubsection{Memory constraints}
	  Specify any applicable characterisitcs and limits on primary and secondary memory.

	  \subsubsection{Operations}
	  Specify special and normal operations required by the user such as:
	  The various modes of operation in the user organization; user initiated operations.
	  Periods of interactive operations and periods of unattended operations.
	  Data processing support functions.
	  Backup and recovery operations.

	  \subsubsection{Site adaptation requirements}
	  Define the requirements for any data or initalized sequences that are specific to a given site, mission or operational mode, ie grid values, safety limits etc.
	  Specify the site or mission related features that should be modified to adapt the software to a particular installation. 

     \subsection{Product functions}
     


\end{document}